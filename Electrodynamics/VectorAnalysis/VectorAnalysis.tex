\documentclass[a4paper]{article}
\usepackage{graphicx}
\usepackage{authblk}
\usepackage{multicol}
\usepackage{multicap}
\usepackage{amssymb}
\usepackage{amsmath}
\usepackage[colorlinks,linkcolor=black]{hyperref}

\newenvironment{define}{
    \textbf{Define: }
}{\par}

\newenvironment{property}{
    \textbf{Property: }
}{\par}

\newenvironment{theorem}{
    \textbf{Theorem: }
}{\par}

\newenvironment{proof}{
    \textbf{Proof: }
}{\par}

\newenvironment{principle}{
    \textbf{Principle: }
}{\par}

\setlength{\textwidth}{474pt}
\setlength{\oddsidemargin}{-7pt}

\title{VectorAnalysis}
\author{S.J. Wang}
\date{\today}

\begin{document}
    \maketitle

    A brief introduction to vector analysis for physics students, lacking precise mathematical proof.

    \newpage

    \tableofcontents

    \newpage

    \section{Vector Algebra}
    \subsection{Tri-vector mixed product}

    Tri-vector mixed product contains dot product(inner product) and cross product (outer product) among three vectors. The result is a scalar.

    \define{for vectors $\vec{a},\vec{b}$ and $\vec{c}$, the mixed product is
    \begin{equation}
        \vec{c}\cdot(\vec{a}\times \vec{b})
    \end{equation}
    }

    The result of $\vec{c}\cdot(\vec{a}\times \vec{b})$ equals to the volume of a parallelepiped defined with these 3 vectors. Hence when the order of $\vec{a},\vec{b}, \vec{c}$ doesn't change, the result is fixed. If the order of 2 vectors is changed, the result counter sign.

    \property{for mixed product:
    \begin{equation}
        \vec{a}\cdot(\vec{b}\times\vec{c}) = \vec{b}\cdot(\vec{c}\times\vec{a}) = \vec{c}\cdot(\vec{a}\times \vec{b}) = -\vec{a}\cdot(\vec{c}\times\vec{b})
    \end{equation}
    }

    

\end{document}